\documentclass{article}
\usepackage{ctex}
\usepackage{multicol} %用于实现在同一页中实现不同的分栏
\usepackage{listings}
\usepackage{ctex}
\usepackage{xcolor}
\usepackage{graphicx}
\usepackage{float}
\usepackage{subfigure}
\usepackage{hyperref}

\setlength{\columnsep}{1cm}

\lstset{
    basicstyle          =   \sffamily,          % 基本代码风格
    keywordstyle        =   \bfseries,          % 关键字风格
    commentstyle        =   \rmfamily\itshape,  % 注释的风格,斜体
    stringstyle         =   \ttfamily,  % 字符串风格
    flexiblecolumns,                % 别问为什么,加上这个
    numbers             =   left,   % 行号的位置在左边
    showspaces          =   false,  % 是否显示空格,显示了有点乱,所以不现实了
    numberstyle         =   \zihao{-5}\ttfamily,    % 行号的样式,小五号,tt等宽字体
    showstringspaces    =   false,
    captionpos          =   t,      % 这段代码的名字所呈现的位置,t指的是top上面
    frame               =   lrtb,   % 显示边框
}
\lstdefinestyle{Python}{
    language        =   Python, % 语言选Python
    basicstyle      =   \zihao{-5}\ttfamily,
    numberstyle     =   \zihao{-5}\ttfamily,
    keywordstyle    =   \color{blue},
    keywordstyle    =   [2] \color{teal},
    stringstyle     =   \color{magenta},
    commentstyle    =   \color{red}\ttfamily,
    breaklines      =   true,   % 自动换行,建议不要写太长的行
    columns         =   fixed,  % 如果不加这一句,字间距就不固定,很丑,必须加
    basewidth       =   0.5em,
}
% In case you need to adjust margins:
\topmargin=-0.45in      %
\evensidemargin=0in     %
\oddsidemargin=0in      %
\textwidth=6.5in        %
\textheight=9.0in       %
\headsep=0.25in         %

\begin{document}
\title{\textbf{编译小组作业报告-C2Python编译器}}
\author{杜婉晴 2021011824 陈植 2021011798 苟芳菲 2021011837}
\maketitle \thispagestyle{empty}
\section{开发环境}

\textbf{编程语言:}Python

\textbf{词法及语法分析工具: }
\href{https://tastones.com/stackoverflow/python-language/python-lex-yacc/getting_started_with_ply/}{Lex-Yacc,PLY库}。

\textbf{代码运行方法: }在官网下载并安装PLY库后,运行translate.py,根据提示输入C语言文件名,翻译结果输出在result.py。

\section{实现功能}
\subsection{词法分析}
1. 识别C语言中的保留字,如类型标识符、控制语句、循环语句等。

2. 忽略“//”注释

3. 识别数字(整数、16进制整数、浮点数)、字符串等常量

4. 识别变量、函数名称

5. 识别其他符号,如算数及逻辑运算符、分隔符、括号等

\subsection{语法分析}
1. 总结C语言文法,能够识别include语句、函数和变量声明及定义、赋值语句、算数及逻辑运算、if-else和switch-case控制语句、for和while循环语句、return跳转语句等。


2. 构建语法分析树,输出为json格式。

\subsection{Python代码生成}
1. 翻译函数声明和定义,自动识别C语言的main函数并翻译为Python程序的入口,函数体内语句能够正确缩进。

2. 翻译include语句,替换为python所需的特定import语句。

3. 翻译C语言的变量声明的定义语句、赋值语句、算数及逻辑运算、if-else和switch-case控制语句、for和while循环语句、return跳转语句等。

4. 翻译C语言中的库函数scanf、printf和strlen,替换为python中的input、print和len函数。


\section{实现原理}
\subsection{词法分析}
词法分析运用了lex工具,通过定义token名称和对应的正则表达式,可识别出C语言中的各种token。并且通过定义保留字列表,避免C语言中的保留字被识别为变量名。

\subsection{语法分析}
语法分析利用了ply.yacc库,通过定义文法规则,可识别出C语言中的各种语句。
我们自行总结了文法规则,大致思路如下:

1.把C语言程序视为外部声明的集合,外部声明包括函数声明和定义、变量声明和定义、include语句等。

2. 函数定义由类型、函数名、参数列表、函数体组成,函数体由语句列表组成。

3. 语句列表由语句组成,语句包括声明语句、赋值语句、表达式语句、选择语句、循环语句、跳转语句等。语句是由每种语句的内容和分号组成

4. 表达式由运算单元和运算符组成,运算单元包含单目运算符和单元表达式,单元表达式包括标识符、常量、函数调用、数组、括号表达式等。赋值语句由标识符、赋值运算符和表达式组成。


\subsection{语法树构建}
1. 定义语法树的节点类,包括叶子节点和非叶子节点。两类节点都有数值属性,其中,叶子节点的值为token的值,非叶子节点的值为节点的类型;非叶子节点还有子节点列表属性。

2. 定义语法树的构建方法,通过递归的方法,将token流转化为语法树。

\subsection{Python代码生成}
目标代码生成部分使用了语法制导的语义处理技术,每个节点用综合属性code来记录该节点及其子树的Python代码,通过递归的方式一遍求出翻译结果。
我们使用code的列表嵌套层数来记录缩进层数,在翻译时遇到需要缩进的代码块时,会将该节点的code列表本身作为一个元素加入到父节点的code列表中,代表缩进层数加一。
code生成完毕后,我们再进行代码风格优化和缩进处理,在字符串间添加空格,并根据嵌套层数添加缩进。

函数定义:去掉函数和参数的类型,加上“def”,遍历函数体中的语句节点并调用对应的翻译函数,把函数体作为一个列表元素加入code中以增加缩进。需要特殊判断函数名是否为main,若是则替换为“if \_\_name\_\_ == '\_\_main\_\_':”语句。

迭代语句:统一翻译成while的形式。对于for循环,翻译成如下形式:
\begin{lstlisting}[language=Python]
    初始化表达式
    while 条件表达式:
        循环体
        更新表达式
\end{lstlisting}
此处要将循环体和更新表达式整体作为一个元素加入code中以增加缩进。对于while循环,直接去掉括号即可。
\section{分工}
杜婉晴:标识符,数字,保留字;定义文法;翻译函数定义、迭代语句、赋值语句

陈植:算数及逻辑运算符,字符串;构建语法分析树;翻译表达式语句、跳转语句

苟芳菲:其他符号;定义文法;翻译变量定义与声明语句、选择语句
\end{document}