\documentclass{article}
\usepackage{ctex}
\usepackage{multicol} %用于实现在同一页中实现不同的分栏
\usepackage{listings}
\usepackage{ctex}
\usepackage{xcolor}
\usepackage{graphicx}
\usepackage{float}
\usepackage{subfigure}
\usepackage{hyperref}

\setlength{\columnsep}{1cm}

\lstset{
    basicstyle          =   \sffamily,          % 基本代码风格
    keywordstyle        =   \bfseries,          % 关键字风格
    commentstyle        =   \rmfamily\itshape,  % 注释的风格,斜体
    stringstyle         =   \ttfamily,  % 字符串风格
    flexiblecolumns,                % 别问为什么,加上这个
    numbers             =   left,   % 行号的位置在左边
    showspaces          =   false,  % 是否显示空格,显示了有点乱,所以不现实了
    numberstyle         =   \zihao{-5}\ttfamily,    % 行号的样式,小五号,tt等宽字体
    showstringspaces    =   false,
    captionpos          =   t,      % 这段代码的名字所呈现的位置,t指的是top上面
    frame               =   lrtb,   % 显示边框
}
\lstdefinestyle{Python}{
    language        =   Python, % 语言选Python
    basicstyle      =   \zihao{-5}\ttfamily,
    numberstyle     =   \zihao{-5}\ttfamily,
    keywordstyle    =   \color{blue},
    keywordstyle    =   [2] \color{teal},
    stringstyle     =   \color{magenta},
    commentstyle    =   \color{red}\ttfamily,
    breaklines      =   true,   % 自动换行,建议不要写太长的行
    columns         =   fixed,  % 如果不加这一句,字间距就不固定,很丑,必须加
    basewidth       =   0.5em,
}
% In case you need to adjust margins:
\topmargin=-0.45in      %
\evensidemargin=0in     %
\oddsidemargin=0in      %
\textwidth=6.5in        %
\textheight=9.0in       %
\headsep=0.25in         %

\begin{document}
\title{\textbf{编译小组作业报告-C2Python编译器}}
\author{杜婉晴 2021011824 陈植 2021011798 苟芳菲 2021011837}
\maketitle \thispagestyle{empty}
\section{开发环境}

\textbf{编程语言:}Python

\textbf{词法分析工具: }
\href{https://tastones.com/stackoverflow/python-language/python-lex-yacc/getting_started_with_ply/}{Lex-Yacc,PLY库}。

\textbf{代码运行方法: }在官网下载并安装PLY库后,直接运行lex.py,默认的输入文件是huiwen.cpp,可以通过命令参数输入文件名。词法分析的token流会输出到output.txt中,格式为LexToken(Token,识别的字符串,行数(未更新),位置)。

\section{实现功能}
\subsection{词法分析}
1. 识别C语言中的保留字,如类型标识符、控制语句、循环语句等。

2. 忽略“//”注释

3. 识别数字(整数、16进制整数、浮点数)、字符串等常量

4. 识别变量、函数名称

5. 识别其他符号,如算数及逻辑运算符、分隔符、括号等

\section{实现原理}
\subsection{词法分析}
词法分析运用了lex工具,通过定义token名称和对应的正则表达式,可识别出C语言中的各种token。并且通过定义保留字列表,避免C语言中的保留字被识别为变量名。

\section{分工}
杜婉晴:标识符,数字,保留字

陈植:算数及逻辑运算符,字符串

苟芳菲:其他符号
\end{document}